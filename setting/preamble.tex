% ===========================
% preamble
% 

\usepackage[twoside,top=22.5truemm,bottom=22.5truemm,left=22.25truemm,right=22.25truemm,dvipdfmx]{geometry} % from naito-san

\usepackage{lipsum}

\usepackage{siunitx}
\usepackage{hhline}
\usepackage{mathrsfs}

% for pdf 
\usepackage{graphicx}
\usepackage{color}

% 
\usepackage{adjustbox}
\usepackage{dcolumn}
\usepackage{bm}% bold math
\usepackage{multirow}
\usepackage{booktabs}
\usepackage{afterpage}
\usepackage{amsmath}
\usepackage{amsthm}
\usepackage{amssymb}
\theoremstyle{plain}
\newtheorem{thm}{Theorem}
\usepackage{ulem}
\usepackage{physics}
\usepackage[nohyperlinks,nolist]{acronym} % abbreviation

%https://stackoverflow.com/questions/2854299/getting-subsection-to-list-in-table-of-contents-in-latex
\setcounter{tocdepth}{2}

% https://mathlandscape.com/latex-underline/
\usepackage{ulem}
\usepackage{here} % force figures Here.
\usepackage{caption}
\usepackage[subrefformat=parens]{subcaption}
%\usepackage{floatrow}
%\usepackage[export]{adjustbox}
\usepackage[version=4]{mhchem}

% for tikz 
\usepackage{tikz, pgf, pgfplots, pgfplotstable}
\usepackage{tikz-3dplot}
\usetikzlibrary{math,calc}
\pgfplotsset{compat = newest}
\usepgfplotslibrary{groupplots} % LATEX and plain TEX
\pgfplotsset{compat = newest}
\usepgfplotslibrary{colorbrewer}
\usetikzlibrary{pgfplots.colorbrewer}

\usepackage{multirow}

\usepackage[subpreambles=true,sort=true]{standalone}

\newcommand{\red}[1]{\textcolor{red}{#1}}
\newcommand{\blue}[1]{\textcolor{blue}{#1}}
\arraycolsep=0.0em
\setlength{\abovecaptionskip}{0mm}
\setlength{\belowcaptionskip}{0mm}
\usepackage{caption} 
\captionsetup[table]{skip=8pt}
\captionsetup[figure]{skip=4pt}
%\setlength{\MidlineHeight}{2pt}


\usepackage{comment}

% for bibtex
\usepackage[square,numbers,sort&compress]{natbib}
%% \usepackage[sectionbib]{chapterbib}  
% \usepackage[natbib,style=phys,refsection=chapter]{biblatex}
% \addbibresource{references/ref.bib}

 % revtex4-2 は natbib を中で usepackage している
 % square : [] で囲む
 % numbers : 通し番号で表現
 % sort&compress : 連続する文献はハイフンでつなぐ

% setting for bib by chapter
\usepackage{bibunits}
\bibliographystyle{spphys} 
\defaultbibliographystyle{spphys} 
\defaultbibliography{references/ref}

% revtex4-2 は natbib を中で usepackage している
% square : [] で囲む
% numbers : 通し番号で表現
% sort&compress : 連続する文献はハイフンでつなぐ

\usepackage[colorlinks=true,citecolor=blue,linkcolor=blue,urlcolor=blue]{hyperref}
% \AddToHook{begindocument/before}{\usepackage[colorlinks=true,citecolor=blue,linkcolor=blue,urlcolor=blue]{hyperref}}
\usepackage{cleveref}
\crefname{equation}{Eq.}{Eq.}% {環境名}{単数形}{複数形} \crefで引くときの表示
\crefname{figure}{Fig.}{Fig.}% {環境名}{単数形}{複数形} \crefで引くときの表示
\crefname{table}{Table}{Table}% {環境名}{単数形}{複数形} \crefで引くときの表示
\crefname{section}{Sec.}{Sec.}% {環境名}{単数形}{複数形} \crefで引くときの表示
\crefname{appendix}{Appendix}{Appendix}% {環境名}{単数形}{複数形} \Crefで引くときの表示

\Crefname{equation}{Equation}{Equation}% {環境名}{単数形}{複数形} \Crefで引くときの表示
\Crefname{figure}{Figure}{Figure}% {環境名}{単数形}{複数形} \Crefで引くときの表示
\Crefname{table}{Table}{Table}% {環境名}{単数形}{複数形} \Crefで引くときの表示
\Crefname{section}{Section}{Section}% {環境名}{単数形}{複数形} \Crefで引くときの表示
\Crefname{appendix}{Appendix}{Appendix}% {環境名}{単数形}{複数形} \Crefで引くときの表示


\newcommand{\ph}{\phantom{0}}

\renewcommand{\topfraction}{1.0}
\renewcommand{\bottomfraction}{1.0}
\renewcommand{\dbltopfraction}{1.0}
\renewcommand{\textfraction}{0.1}
\renewcommand{\floatpagefraction}{0.9}
\renewcommand{\dblfloatpagefraction}{0.9}









