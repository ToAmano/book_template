%
% 2023/11/1
% 
% 参考文献
% https://molina.jp/blog/tikz%E3%81%A6%E3%83%95%E3%83%AD%E3%83%BC%E3%83%81%E3%83%A3%E3%83%BC%E3%83%88%E3%82%92%E6%9B%B8%E3%81%8F/
% https://tikz.dev/library-shapes
% https://tex.stackexchange.com/questions/42611/list-of-available-tikz-libraries-with-a-short-introduction


\documentclass[varwidth=\linewidth,border=0.5cm,tikz]{standalone}
% https://latexdraw.com/plot-a-function-and-data-in-latex/
 \usepackage[ipaex]{luatexja-preset} % standaloneとlualatexjaを共存させる方法?
\usepackage{tikz}
\usepackage{pgfplots}
\usepackage{physics}

\pgfplotsset{compat = newest}
 % axis style, ticks, etc
 \pgfplotsset{every axis/.append style={
                    label style={font=\Huge}, % ラベルだけ大きく
                    title style={font=\huge},
                    tick label style={font=\huge},
                    legend style={font=\huge},
                    }}
 \usepgfplotslibrary{colorbrewer}
 \usetikzlibrary{pgfplots.colorbrewer}
\usetikzlibrary{arrows.meta,backgrounds}
\tikzset{white background/.style={show background rectangle,tight background,
background rectangle/.style={fill=white}}}

% workflow 
% https://qiita.com/danielBEE/items/77571f2d987cba84f580
\usetikzlibrary{shapes,arrows,positioning}
% https://tex.stackexchange.com/questions/13030/node-shapes-in-tikz
% https://tikz.dev/library-shapes
\usetikzlibrary{shapes,shapes.multipart,shapes.symbols}

\usepackage{}	% required for `\align' (yatex added)
\begin{document}


\begin{tikzpicture}
  % 図形の定義
% \node[name=s, shape=signal, signal from=west, shape example, inner sep=2cm]
  \tikzset{title/.style={rectangle, draw, text centered, text width=20cm, minimum height= 2.5cm,font=\huge}};
  \tikzset{flow/.style={signal, signal from=west, draw, text centered, text width=7cm, minimum height=1.5cm,font=\huge}};
  \tikzset{nonflow/.style={signal, signal from=west, draw=none, text centered, text width=7cm, minimum height=1.5cm}};
  \tikzset{nonblock/.style={rectangle, draw=none, text centered, text width=3cm, minimum height= 1.5cm}}; %
  \tikzset{block/.style={rectangle, draw, text centered, text width=3cm, minimum height= 3cm,font=\huge}};
  \tikzset{process/.style={rectangle, draw, text centered, text width=8cm, minimum height= 3cm, font=\Large}};
  \tikzset{nonprocess/.style={rectangle, draw=none, text centered, text width=8cm, minimum height= 3cm}};
%  \tikzset{cloud/.style={rounded rectangle, draw, text centered, text width=3cm, minimum height=1.5cm}};
%  \tikzset{decision/.style={diamond, draw, text centered, aspect=3,text width=5cm, minimum height=1.5cm}};
  \tikzset{line/.style={draw, very thick, color=black!50, -latex'}} % 線
  %placenode

  \node[title](-1){Dielectric function 
\begin{align*}
 \varepsilon_{\alpha\beta}(\omega)&=\varepsilon^{\infty}_{\alpha\beta}+\frac{1}{v_0}\sum_{j}\frac{\textcolor{Set1-C}{S^{j}_{\alpha\beta}}}{\left(\textcolor{Set1-A}{\Omega^{\mathrm{SCPH+B}}_{\vb*{0}j}}\right)^2-\omega^2-2\textcolor{Set1-A}{\Omega^{\mathrm{SCPH+B}}_{\vb*{0}j}}\textcolor{Set1-B}{\Sigma^{\mathrm{B+4ph}}(\omega)} }
\end{align*}
}; 
  % titleと他の部分の調整
  \node[nonflow, below= of -1](-2){};
  \node[nonflow, left= of -2](-3){};

 % 以下が実際の部分
  \node[nonblock, left= of -3](0){};
  \node[block, below= of 0](1){Real part of $\Sigma$};
  \node[block, below= of 1](2){Imaginary part of $\Sigma$};
  \node[block, below= of 2](3){Mode-oscillator strength};
  \node[flow, right= of 0](4){SCPH};
  \node[flow, right= of 4](5){SCPH+B};
  \node[flow, right= of 5](6){Post-process};
  \node[process, below= of 4](7){Consider Tadpole and Loop \\ Get SCPH frequencies $\omega^{\mathrm{SCPH}}$};
  \node[process, below= of 5](8){Consider Real part of Bubble \\ Get SCPH+B frequencies $\textcolor{Set1-A}{\Omega^{\mathrm{SCPH+B}}}$};
  \node[nonprocess, below= of 6](9){};
  \node[process, below= of 9](10){Get imaginary part of Bubble and 4ph 
     \begin{align*}
   \textcolor{Set1-B}{\Sigma^{\mathrm{B+4ph}}(\omega)} =&\Im \Sigma^{\mathrm{B}}[G^{\rm{SCPH+B}},\Phi_3](\omega) \\ 
     + &\Im \Sigma^{\mathrm{4ph}}[G^{\rm{SCPH+B}},\Phi_4](\omega)
   \end{align*}};
  \node[process, below= of 10](11){Use Born effective charges and SCPH+B eigenvectors to get $\textcolor{Set1-C}{S}$};


  %Drawedges(矢印を指定する。直線は--で繋げて、直角線は-|でつなげる。)
  % --もしくは-|の後に、 node[オプション]{矢印に沿って書きたい文字}を書ける。
  \path[line](7)--(8);
  \path[line](8)|-(10);
  \path[line](8)|-(11);
  
  % \path[line](1)-| node[below, near start, color=black]{Yes}(3);
\end{tikzpicture}
\end{document}

